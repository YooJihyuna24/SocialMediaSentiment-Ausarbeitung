%%%%%%%%%%%%%%%%%%%%%%%%%%%%%%%%%%%%%%%%%%%%%%%%%%%%%%%%%%%%%%%%%%%%%%%%%%%%%%%
%% 2020-06-20
%% Descr:       Variablen für die Projektarbeit / Bachelorarbeit festlegen
%% Author:      Vorlage erstellt von Daniel Spitzer an der DHBW Lörrach 
%% Angepasst:   Katja Wengler, ZWI, DHBW Karlsruhe
%%%%%%%%%%%%%%%%%%%%%%%%%%%%%%%%%%%%%%%%%%%%%%%%%%%%%%%%%%%%%%%%%%%%%%%%%%%%%%%

% Hier müssen die Variablen zur eignenen Arbeit angepasst werden

% Der Titel der Arbeit, der auf dem Deckblatt angezeigt wird
\def \thesisTitle {Sentiment-Analyse der psychischen Gesundheit mit CRISP-DM}

% Der Titel der Arbeit, der in der Fußzeile angezeigt wird
\def \thesisFooterTitle {}

% Art der Arbeit
\def \thesisType {Projektarbeit DSKI}

% Bildungsabschluss
\def \degree {Bachelor of Science (B. Sc.)}

% Abgabedatum
\def \submissionDate {25. April 2025}

% Studiengang
\def \courseOfStudies {Data Science und Künstliche Intelligenz}

% Kurs
\def \course {WDS24B1}

% Name des Autors der Arbeit
\def \name {Björn Hilgers, Celine Lagler, Jihyun Yoo, Robbie Sumner}

% Name der Ausbildungsfirma
\def \company {}

% Ort, an dem Ausbildungsfirma ansässig ist
\def \companyLocation {}

% Betreuer der Ausbildungsfirma
\def \corporateAdvisor {Niklas Lederer, Stefan Eckerle}

% Wissenschaftlicher Betreuer
\def \universityAdvisor {}

% Ort und Datum für die Ehrenwörtliche Erklärung
\def \declarationHeading {Selbstständigkeitserklärung}
\def \declarationLocation {Karlsruhe}
\def \declarationDate {8. Dezember 2024}

% Ort und Datum für die Freigabe der Arbeit
\def \releaseLocation {Karlsruhe}
\def \releaseDate {8. Dezember 2024}

% Name der Bilddatei für das Firmenlogo. Die Datei muss im Ordner images sein.
% Erlaubt sind u.a. folgende Formate: PDF, PNG, JPEG
\def \fileNameLogo {company_logo.pdf}

% Je nach Länge des Titels kann die Schriftgröße angepasst werden
\def \titleFontSize {18}		% Schriftgröße des Titels auf dem Deckblatt
\def \footerFontSize {10}		% Schriftgröße in der Fußzeile

% Wenn es sich um eine Seminararbeit handelt,
% muss der folgende Befehl durch diesen ersetzt werden:
\seminararbeittrue
% \seminararbeitfalse

% Wenn die Arbeit keinen Sperrvermerk hat,
% muss der folgende Befehl durch diesen ersetzt werden:
\blockingnoticefalse
% \blockingnoticetrue

% Die Daten für den Sperrvermerk. Diese müssen natürlich nur geändert werden,
% wenn die Arbeit einen Sperrvermerk hat.
% Datum der Unterschrift des Autors auf dem Sperrvermerk
\def \blockingNoticeAuthorDate {31. August 2020}
% Datum der Unterschrift des Unternehmensvertreters auf dem Sperrvermerk
%\def \blockingNoticeCompanyDate {31. August 2020}

% Adresse und Land des Unternehmens auf dem Sperrvermerk. Die \\ sorgen für einen Zeilenumbruch
\def \companyAdress {Hauptstraße 1 \\ 76133 Karlsruhe \\ Deutschland}
% Telefonnummer des Unternehmens auf dem Sperrvermerk.
\def \companyPhone {0721 1234567}
% E-Mailadresse des Unternehmens auf dem Sperrvermerk.
\def \companyEmail {info@musterfrau-ag.de}
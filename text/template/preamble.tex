%%%%%%%%%%%%%%%%%%%%%%%%%%%%%%%%%%%%%%%%%%%%%%%%%%%%%%%%%%%%%%%%%%%%%%%%%%%%%%%
%% 2020-06-20
%% Descr:       Formale Einstellungen für die Praxisarbeit
%% Author:      Vorlage erstellt von Daniel Spitzer an der DHBW Lörrach 
%% Angepasst:   Katja Wengler, ZWI, DHBW Karlsruhe
%%%%%%%%%%%%%%%%%%%%%%%%%%%%%%%%%%%%%%%%%%%%%%%%%%%%%%%%%%%%%%%%%%%%%%%%%%%%%%%

\documentclass[a4paper,12pt]{article}
\usepackage{fontspec}
\usepackage{polyglossia}
\usepackage{csquotes}
\setdefaultlanguage[spelling=new, babelshorthands=true]{german}

% Seitenränder
\usepackage[left=3.5cm, right=2.5cm, head=1.25cm, bottom=2cm, foot=1.25cm, includefoot]{geometry}

% Abkürzungsverzeichnis
\usepackage[printonlyused]{acronym}

% Gut formatierte Tabellen
\usepackage{tabulary}
\usepackage{tabularx}  
\usepackage{array}

% Positioniert Tabellen und Abbildungen
\usepackage{float}

% Fügt Abschnitte wie den Anhang oder die Kurzfassung dem Inhaltsverzeichnis hinzu
\usepackage{tocbibind}

% Ermöglicht Abbildungen in LaTeX
\usepackage{graphicx}
\graphicspath{ {images/} }

% Definiert die Farben für Tabellen im DHBW-Style
\usepackage[table]{xcolor}
\definecolor{tableHeading}{gray}{0.672}
\definecolor{tableOdd}{gray}{0.945}
\definecolor{tableEven}{gray}{0.859}
\arrayrulecolor{white}

% Schriftart Carlito. Sie ist fast identisch mit Calibri.
% Calibri ist auf Linux und MacOS nicht verfügbar. 
\usepackage{carlito}
\setmainfont{carlito}

% Fußzeile und Kopfzeile
\usepackage{fancyhdr}
\renewcommand{\headrulewidth}{0pt}
\renewcommand{\footrulewidth}{0.5pt}
\def \footer{
{\centering
\fontsize{\footerFontSize}{\footerFontSize}\selectfont%\thesisFooterTitle
\thepage\par}
}



% Positioniert die Fußnoten fest am unteren Ende der Seite
\usepackage[bottom]{footmisc}

% Zeilenabstand: 1.5
\usepackage{setspace}
\setstretch{1.5}

% Literaturverzeichnis
\usepackage[natbib=true, backend=biber, style=authoryear, dashed=false]{biblatex}
\DeclareBibliographyAlias{interview}{misc}
\addbibresource{text/bibliography.bib}

% Das Paket hyperref muss als letztes Paket geladen werden. Das Paket setzt Links im PDF-Dokument.
\usepackage[hidelinks, unicode]{hyperref}
\hypersetup{pdftitle = {\thesisTitle}, pdfauthor = {\name}}

% Definiert die Farben für Code-Listings
\usepackage{xcolor}
\definecolor{RoyalBlue}{rgb}{0.25, 0.41, 0.88}
\definecolor{BrickRed}{rgb}{0.8, 0.25, 0.33}
\definecolor{OliveGreen}{rgb}{0.33, 0.42, 0.18}

% Code-Listings für Python
\setmonofont{sourcecodepro}
\usepackage{listings}
\lstset{
    basicstyle=\ttfamily\scriptsize, % Grundstil: Monospace, kleine Schrift
    frame=single, % Rahmen um den Code
    language=Python, % Programmiersprache: Python
    keywordstyle=\color{RoyalBlue}, % Schlüsselwörter: Weniger intensives Blau
    stringstyle=\color{BrickRed}, % Zeichenketten: Weniger intensives Rot
    commentstyle=\color{OliveGreen}, % Kommentare: Weniger intensives Grün
    showstringspaces=false, % Leerzeichen in Zeichenketten nicht anzeigen
    numbers=left, % Zeilennummern links
    numberstyle=\tiny\color{gray}, % Stil der Zeilennummern: Klein, grau
    breaklines=true, % Zeilenumbruch bei langen Zeilen
    captionpos=b, % Position der Beschriftung: Unten
}

% Normale Schriftart für URLs
\renewcommand{\UrlFont}{}

% Variable für das Speichern der Seitenzahl (römisch -> arabisch -> römisch)
\newcounter{pageNumber}

% Nummerierung: 2.1, 2.2 usw.
\renewcommand{\labelenumii}{\theenumii}
\renewcommand{\theenumii}{\theenumi.\arabic{enumii}.}

% Bei Bedarf können die Überschriften geändert werden
\def \declarationHeading{Selbstständigkeitserklärung}
%\def \thesisSizeHeading{Hinweis zum Umfang der Arbeit}
%\def \releaseHeading{Freigabe der Arbeit}
\def \blockingHeading{Sperrvermerk}
\def \abstractHeading{Kurzfassung}
\def \appendixHeading{Anhang}
\def \referenceHeading{Quellenverzeichnis}
\def \acronymHeading{Abkürzungsverzeichnis}

% Befehl für ein einleitendes Zitat
\newcommand{\epigraph}[2]{
    \begin{quote}\begin{quote}
        \begin{center}
            \textit{#1}
        \end{center}
        \hfill #2
    \end{quote}\end{quote}
}
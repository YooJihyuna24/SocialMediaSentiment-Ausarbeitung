\sloppy
\section{Zielsetzung des Projekts anhand von SMART-Zielen}
Das Ziel dieses Projekts ist es, ein interaktives Dashboard zu entwickeln, das Sentiment-Analysen anhand von Social Media Daten durchführt.
Spezifisch erzeugen wir mit dem Python-Modul "streamlit" ein Dashboard, und beziehen Daten von der Social Media Plattform "Reddit" mit Hilfe der API "PRAW".
Für die Analyse wird ein vortrainiertes Machine-Learning-Modell verwendet, dass wir von "HuggingFace" beziehen.
Der Nutzer hat die Möglichkeit, zwischen verschiedenen Modellen auszuwählen und die Ergebnisse der Sentiment-Analyse in einem interaktiven Dashboard zu visualisieren. 
Die Ergebnisse werden dann in einem Dashboard visualisiert, um den Nutzern eine einfache und intuitive Möglichkeit zu bieten, die Sentiment-Analysen zu verstehen und zu interpretieren.
\\
Das Projekt ist erfolgreich, wenn die folgenden Ziele erreicht sind:
\begin{enumerate}
    \item \textbf{Fehlerfreies und funktionsfähiges Dashboard: } \\
    Das Dashboard führt erfolgreich die Sentiment-Analyse durch und visualisiert die Ergebnisse korrekt. Das Dashboard ist modular und benutzerfreundlich aufgebaut.
    \item \textbf{Verwendung fortgeschrittener Konzepte der Programmierung:} \\
    Der Code wird gemäß fortgeschrittener Programmierkonzepte strukturiert. Dies umfasst:

    \begin{itemize}
        \item \textbf{Ausnahmebehandlung und Unittests: } \\
        Es werden geeignete Mechanismen zur Fehlerbehandlung implementiert, um Laufzeitfehler zu vermeiden und eine robuste Anwendungslogik zu gewährleisten.
    \end{itemize}
    \newpage

    \item \textbf{Nutzung von Machine Learning Modellen: } \\
    Für die Sentiment-Analyse wird ein vortrainiertes Modell von HuggingFace verwendet.
    Verschiedene Modelle werden evaluiert und der Nutzer kann zwischen diesen Modellen auswählen.
    Dadurch wird eine flexible und anpassbare Sentiment-Analyse ermöglicht.
    Zusätzlich erfolgt eine Beurteilung der Effizienz der Modelle basierend auf ihren Ergebnissen und Laufzeiten.

    \item \textbf{Dokumentation:} \\
    Das Projekt wird unter Verwendung von Git versioniert und im Git-Repository dokumentiert, inklusive einer README-Datei, die Installations- sowie Nutzungshinweise gibt.
\end{enumerate}\\
Das Projekt ist realistisch innerhalb des gegebenen Zeitrahmens durchführbar.
Es erfordert fortgeschrittene Programmierkenntnisse und den Umgang mit gängigen Data-Science- und Machine-Learning-Bibliotheken, was im Rahmen des Moduls:
\\
"Fortgeschrittene Programmierung" zu erlernen und umzusetzen ist.
Die Abgabe des Projekts erfolgt am \submissionDate.
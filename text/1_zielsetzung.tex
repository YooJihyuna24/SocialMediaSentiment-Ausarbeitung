\section{Einleitung}

\subsection{Ausgangssituation und Motivation}
Die zunehmende Bedeutung sozialer Medien als Meinungsplattformen birgt Potenziale für datengetriebene Analysen öffentlicher Stimmungen. Unternehmen und Organisationen benötigen Werkzeuge, um relevante Diskussionen frühzeitig zu erkennen und einzuordnen. Technisch erfordert dies robuste Schnittstellen zu Plattformen wie Reddit sowie Methoden zur automatisierten Sprachverarbeitung (\emph{NLP} = Natural Language Processing).

\subsection{Projektidee und Anwendungsfall}
Ziel dieses Projekts ist die Entwicklung eines webbasierten Dashboards zur Analyse von Sentiment-Daten aus Reddit-Foren. Als Use Case dient die kontinuierliche Beobachtung gesellschaftlicher Diskurse, etwa in Bereichen wie Mental Health oder Produktfeedback. Die Anwendung richtet sich an Forschende, Journalist:innen oder Organisationen mit Interesse an digitalen Meinungsbildern.

\subsection{SMART-Ziele}
Das Projekt verfolgt klar definierte SMART-Ziele:
\begin{itemize}
    \item \textbf{Specific:} Entwicklung eines interaktiven Dashboards mit API-Anbindung und Sentimentanalyse.
    \item \textbf{Measurable:} Integration und Vergleich von Machine Learning-Modellen zur Sentimentanalyse; Visualisierung von Key Performance Indicators (KPIs).
    \item \textbf{Achievable:} Bereitstellung einer benutzerfreundlichen Oberfläche.
    \item \textbf{Realistic:} Umsetzung im Rahmen eines studentischen Projekts.
    \item \textbf{Time-Bound:} Abgabe der lauffähigen Anwendung mit umfassender Dokumentation bis 25. April 2025.
\end{itemize}
\section{Ergebnisse und Funktionalität}

Das zentrale Ergebnis des Projekts ist ein funktionierendes, interaktives Web-Dashboard, das mithilfe von Streamlit realisiert wurde. Es ermöglicht Nutzern die Sentiment-Analyse von Inhalten der Plattform Reddit, sowohl auf der Ebene von Subreddits als auch für einzelne Posts.

\subsection{Vorstellung des finalen Dashboards}

Die Anwendung präsentiert sich dem Nutzer mit einer übersichtlichen Oberfläche, die im Wesentlichen aus zwei Hauptbereichen besteht: einer Seitenleiste \verb|st.sidebar| für globale Einstellungen und einem Hauptbereich zur Darstellung der Analyseergebnisse.

\subsubsection{Vorstellung der Seitenleiste}

Die Seitenleiste dient der globalen Steuerung. Hier kann der User zwischen den zwei Hauptansichten \enquote{Subreddit-Dashboard} und \enquote{"Posts-Dashboard"} wechseln. Des weiteren kann der User die Analyseparameter konfigurieren. Mittels des Steuerelements \newline
\verb|st.pills| kann zwischen den Sentiment-Analyse-Modellen gewählt werden - dem DistilBERT, DistilRoBERTa und RoBERTa. Zusätzlich gibt es noch die \enquote{pills} multilingual und political für eine potenziell spezifischere Analyse.
Ebenfalls kann über das Steuerelement \verb|st.pills| gefiltert werden, welche Submissions analysiert werden sollen - \enquote{hot} , \enquote{top}, \enquote{new} oder \enquote{rising}. Der \verb|st.select_slider| erlaubt die Einstellung der maximalen Anzahl an Submissions (im Subreddit-Dashboard) bzw. Kommentaren 
(im Posts-Dash-
\newline
board), die in die Analyse einbezogen werden sollen. Hierbei kann die Anzahl zwischen 1 bis 100 ausgewählt werden.

\subsubsection{Vorstellung der Dashboard-Hauptseite}

Nach Eingabe eines Subreddit-Namens im Eingabefeld der Subreddit Dashboard Seite, ruft diese Ansicht über den \verb|data_processor| die relevanten Daten ab. Dem User werden Metriken wie die Abonnentenzahl des Subreddits und der \enquote{hot}-Post angezeigt. Kernstück der Seite ist die Sentiment-Analyse der ausgewählten Submissions. Die Verteilung der erkannten Sentiments (z.B. joy, anger, sadness, fear - je nach Modell) wird aggregiert und als Diagramm (pie-chart) dargestellt. Zudem wird eine WordCloud generiert, die häufig vorkommende Begriffe in den analysierten Submission-Texten visualisiert, nachdem diese bereinigt wurden. 
Das Posts Dashboard fokussiert sich hingegen, nach Eingabe der URL eines spezifischen Reddit-Posts, auf die Diskussion unter diesem Post. Sie zeigt den Score (Upvotes - Downvotes) des Posts an und bettet den Post selbst mithilfe von \verb|visual_helpers.get_reddit_ebed_url| direkt in das Dashboard ein, um Kontext zu bieten. Analog zum Subreddit Dashboard wird die Sentiment-Verteilung der Top-Kommentare analysiert, aggregiert und visualisiert. Die WordCloud für die Kommentar-Texte wird ebenfalls angezeigt.

\subsection{Analysefähigkeiten und Nutzerinteraktion}

Mit dem entwickelten Dashboard kann der Nutzer schnell Einblicke in die Stimmungslage innerhalb spezifischer Reddit-Communities oder deren Diskussionen erhalten. Die wichtigste Funktion umfasst dabei die Erfassung des Stimmungsbilds: Nutzer können die dominanten Emotionen in einem Subreddit oder in der Kommentarsektion eines Posts identifizieren. Durch die Möglichkeit, zwischen den verschiedenen Modellen zu wechseln, können User untersuchen, ob sich die Einschätzung der Sentiment-Analyse je nach verwendetem Modell unterscheidet und das Modell wählen, welches ihren Anwendungsfall am besten unterstützt. Dies ist vor allem dann relevant, wenn die Konnotation von Begriffen kontextabhängig ist (z.B. im politischen Diskurs).
Ebenso können Nutzer durch die Auswahl verschiedener Filter (\enquote{hot}, \enquote{top}, \enquote{new}, \enquote{rising}) die Stimmungslage für unterschiedliche Kategorien von Posts untersuchen, was Hinweise auf aktuelle oder populäre Themen und deren Emotionen geben kann. Die Kombination aus Texteingabefeldern, für das Subreddit Dashboard, Auswahlmöglichkeiten (Modell, Filter) und dem Slider für die Datenmenge erlaubt eine flexible Analyse nach den Bedürfnissen des Users. Zudem helfen die Metriken wie Abonnentenzahl und Post-Score sowie die direkte Einbettung des Posts, die Analyseergebnisse besser einzuordnen.
Die WordCloud bietet zusätzliche qualitative Einblicke in die diskutierten Themen. Nutzer können beispielsweise untersuchen, wie in einem Gaming-Subreddit über ein neues Spiel diskutiert wird (Emotionen in \enquote{new} Posts), die allgemeine Stimmung in einem Finanz-Subreddit analysieren oder vergleichen, wie die Kommentare unter einem politischen Nachrichtenpost von den beiden verschiedenen Modellen bewertet werden.

 
Zusammenfassend liefert das Projekt ein funktionierendes Analysewerkzeug, das die im Code definierten Features und Interaktionsmöglichkeiten bereitstellt und durch gezielte technische Lösungen auf typische Herausforderungen bei der Verarbeitung von Social-Media-Daten und dem Einsatz von ML-Modellen reagiert.

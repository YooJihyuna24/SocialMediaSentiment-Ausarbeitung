\section{Erläuterung der Projektschritte}
\subsection{Projektplanung und Management}

Angesichts der verschiedenen technologischen Komponenten und der Notwendigkeit einer koordinierten Teamarbeit war ein strukturiertes Projektmanagement für den Erfolg des Projekts zur Entwicklung eines Reddit-Sentiment-Analyse-Dashboards unerlässlich. Ziel war es, die Zusammenarbeit effizient zu gestalten, den Fortschritt transparent zu machen und die Einhaltung der Projektziele sicherzustellen. 

Zu Beginn des Projekts \textcolor{yellow}{haben wir uns/ wurde sich} zusammengesetzt und Brainstorming betrieben um auf eine Idee für ein Projekt zu kommen. Dabei wurden verschiedene Ansätze durchdacht und ChatGPT für präzisere Vorschläge verwendet. Problem dabei war es sich für eine Richtung zu entscheiden, \textcolor{yellow}{in die es gehen sollte}. Nachdem wir uns geeinigt haben, welches Projekt wir angehen wollen, ging es zügig voran mit dem Brainstorming. Dazu wurde, \textcolor{yellow}{wie in Bild x zu sehen}, ein erster Entwurf erstellt, indem die Ideen zur Visualisierung gesammelt worden sind. Dies diente der weiteren Planung, um das Projekt in die verschiedenen Hauptkomponenten aufteilen zu können. Damit die anstehenden Aufgaben und die fortschritte auch festgehalten werden können, wurde Notion als Projektmanagementtool verwendet. Notion ist eine webbasierte All-in-One-Software, die der Verwaltung von Informationen, die Funktion wie Notizen, Datenbanken, Projektmanagement und Kollaborationen in einer zentralen Oberfläche vereint. (Quelle GPT Beschreibung?!) Für dieses Projekt wurde insbesondere ein Kanban-Board verwendet, um eine übersichtliche Aufgabenverteilung und Fortschrittskontrolle zu gewährleisten. Dieses Board wurde nach klassischen agilen Prinzipien aufgebaut und umfasste Spalten wie „Backlog“, „To Do“, „In Progress“, „Review“ und „Done“. Jede Aufgabe wurde mit einer kurzen Beschreibung als Karte auf dem Board angelegt. Als nächstes wurden diese Priorisiert und die Aufgaben entsprechend aufgeteilt. Durch das verschieben der Karten konnte somit der Fortschritt der einzelnen Aufgaben jederzeit logisch nachhvollzogen werden und engpässe konnten rechtzeitig erkannt  und gemeinsam gelöst werden.

Zur gemeinsamen Entwicklung des Python Codes wurde die öffentliche Softwareentwicklungsplattform GitHub mit der Versionsverwaltungs-Software Git verwendet. Mit Hilfe dieser ist es leicht Änderungen am Code nachzuvollziehen, die verschiedenen Entwicklungsstände zu verwalten und Konflikte bei gleichzeitigem Arbeiten zu minimieren. GitHub wurrde somit als zentrales Repository verwendet, das als „Single Source of Truth“ für unseren Code diente und gleichzeitig als Backup fungierte. Um paralleles Arbeiten an verschiedenen Funktionalitäten (z.B. API-Anbindung vs. UI-Elemente) zu ermöglichen und die Stabilität des Haupt-Codes sicherzustellen, wurde ein Feature-Branch-Workflow verfolgt. Dabei wurde für jede neue Funktion oder größere Änderung ein eigener Branch erstellt. Nach Fertigstellung und interner Abstimmung wurde der Code des Branches über einen Merge-Prozess wieder in den Hauptentwicklungszweig integriert. Dieses Vorgehen förderte eine strukturierte Entwicklung und erleichterte die Integration der Beiträge aller Teammitglieder.

Ergänzend zu den genannten Tools fanden wöchentlich, meist Online, Teambesprechungen statt. Die Besprechungen dienten dem direkten Austausch der einzelnen Fortschritte der aufgeteilten Aufgaben, der klärung von Fragen, sowie der gemeinsamen Planung der nächsten Schritte. Somit konnte mit den bereits beschriebenen Tools eine effiziente Abstimmung gewährleistet werden. 

Zusammenfassend lässt sich sagen, dass die Kombination aus agiler Aufgabenverwaltung mit Notion, systematischer Versionskontrolle mit Git/GitHub für Code und Dokumentation sowie regelmäßiger Austausch eine solide Grundlage für die erfolgreiche Durchführung dieses Projekts bildete.

\subsection{Einsatz von Streamlit zur Visualisierung}

Für die Entwicklung der grafischen Benutzeroberfläche wurde Streamlit gewählt. Dabei handelt es sich um ein Open-Source-Framework in Python, das es ermöglicht, schnell und einfach interaktive Webanwendungen für Data-Science-Projekte zu erstellen. Die Entscheidung fiel auf Streamlit insbesondere aufgrund der einfachen Integration in den Python-Workflow. Zudem benötigt Streamlit keine zusätzliche Frontend-Programmierung in HTML, CSS und JavaScript. Dies erlaubt eine schnelle Entwicklung interaktiver Prototypen direkt aus Python heraus. 

Mit der Funktion \verb|@st.set_page_config()| wird in der main.py die Initialkonfiguration der Webanwendung vorgenommen. Sie wird am Anfang des Codes aufgerufen und ermöglicht es die grundlegenden Eigenschaften der Webanwendung festzulegen. \\ % wie oft die doku einbinden von streamlit ? % 
Ein weiteres wichtiges Element ist die durch \verb|@st.navigation()| implementierte Seitenstruktur. Diese erlaubt es, das Subreddit-Dashboard und das Posts-Dashboard als separate Seiten in die Anwendung zu integrieren. Durch diese modulare Struktur bleibt die Anwendung auch bei wachsender Funktionalität benutzerfreundlich und übersichtlich.
Entscheidend für die Performance der Streamlit-Anwendung ist der Einsatz mit den Caching-Funktionen \verb|@st.cache_data & st.cache_ressource|. Da Streamlit das gesamte Skript bei jeder Nutzerinteraktion potenziell neu ausführt, würden ohne Caching rechenintensive Operationen wie API-Abfragen oder das Laden von Machine-Learning-Modellen wiederholt durchgeführt. Somit werden die Caching-Funktionen von Streamlit in dem Projekt genutzt, um die Ergebnisse von Aufrufen zwischenzuspeichern. Dies verringert nicht nur die Latenz, sondern schont zudem die API-Rate-Limits von Reddit.

Im Gegensatz zu anderen Dashboardsystemen wie Dash oder Bokeh ist der initiale Konfigurationsaufwand bei Streamlit deutlich geringer, wodurch es sich besonders gut für dieses Projekt eignet. (zitat heaven2021) 
Ein weiterer Vorteil ist die stetig ansteigende Anzahl an Community-Komponenten und Drittanbieter-Integration, wodurch sich auch komplexere Visualisierungen (z.\,B.\ über Plotly oder WordCloud) nahtlos einbinden lassen, wie im weiteren Verlauf des Projekts zu sehen sein wird. 

Diese Merkmale machen Streamlit zu einem geeigneten Werkzeug für datengestützte Projekte, bei denen der Fokus auf schneller Umsetzung und einfacher Bedienbarkeit liegt.
